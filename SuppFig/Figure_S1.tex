\documentclass{report}

\usepackage{graphicx}
\usepackage[top=1in, bottom=1.25in, left=0.5in, right=1.25in]{geometry}
% textcomp package and marvosym package for additional characters
\usepackage{textcomp,marvosym}

\usepackage[percent]{overpic}

\begin{document}


{\bf Figure S1. Dataset quality plots} (A) archaeal and (B) bacterial species used in the analysis are plotted in the GC, OGT (optimal growth temperature) plane to demonstrate uniform coverage and wide range of both GC and OGT values. Uniform coverage is also preserved when considering organisms with codon usage selection (CUS) only (red dots). Archaeal ``dip'' around 60\textcelsius\ and bacterial peak around 37\textcelsius\ are clearly visible. GC and OGT histograms are presented next to the corresponding axis. 

% \begin{figure}[h!]
% 	\centering
% 	% \includegraphics[scale=0.99]{../Publication/SuppFig1_arch_trop.pdf}
% 	\includegraphics[width=0.9\textwidth]{../Publication/SuppFig1_arch_trop.pdf}
% \end{figure}
%
% \begin{figure}[h!]
% 	\centering
% 	% \includegraphics[scale=0.99]{../Publication/SuppFig1_arch_trop.pdf}
% 	\includegraphics[width=0.9\textwidth]{../Publication/SuppFig1_bact_trop.pdf}
% \end{figure}


\begin{center}
\begin{overpic}[width=\textwidth]{../Publication/SuppFig1_arch_trop.png}
\put(83,88){\huge{\bf (A)}}
\end{overpic}
\end{center}

\begin{center}
\begin{overpic}[width=\textwidth]{../Publication/SuppFig1_bact_trop.png}
\put(83,88){\huge{\bf (B)}}
\end{overpic}
\end{center}



\end{document}