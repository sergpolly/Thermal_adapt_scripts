\documentclass{report}

\usepackage{graphicx}
\usepackage[top=0.8in, bottom=0.8in, left=0.5in, right=0.8in]{geometry}

\usepackage[percent]{overpic}

\begin{document}

{\bf Figure S4. Distributions of average CAI compared for organisms with CUS (Codon Usage Selection) and organisms without CUS.}
(A) Distributions of proteome--wide mean CAI for archaea with CUS (red) and non--CUS archaea (blue).
(B) Distribution of proteome--wide mean CAI for bacteria with CUS (red) and non--CUS bacteria (blue). Presented distributions demonstrate compatibility between CUS criteria proposed by Botzman and Margalit and the one used in the present paper.

% \begin{figure}[h!]
% 	\centering
% 	\includegraphics{../Publication/SuppFig4.pdf}
% \end{figure}



\begin{center}
% \begin{overpic}[grid,tics=20,width=\textwidth]{../Publication/Supp9_quintiles.png}
\begin{overpic}[width=\textwidth]{../Publication/SuppFig4.png}
% \put(32,74){\includegraphics[scale=.3] {busstop.mps}}
\put(8,42){\LARGE{\bf A}}
\put(56,42){\LARGE{\bf B}}
	% \put(32,77){\llap{\scriptsize \colorbox{back}{Windm\"uhle}}}
% \put(28,63){\small\textcolor{red}{\ding{55}}}
% \put(17.5,11){\scriptsize\colorbox{back}{{\Pisymbol{ftsy}{65} Fr}}}
% \put(6.3,13){\colorbox{back}{{\Pisymbol{ftsy}{68}}}}
% \put(29.8,61.4){\color{blue}\vector(-1,-3){2}}
% \put(38.6,63){\color{blue}\vector(1,3){2}}
\end{overpic}
\end{center}



\end{document}