\documentclass{report}

\usepackage{graphicx}
\usepackage[top=0.8in, bottom=0.8in, left=0.5in, right=0.8in]{geometry}

\begin{document}

{\bf Figure S5. Simulated frequencies of amino acids compared with the naturally evolved ones for archaea.} Pearson correlation coefficient is used to compare simulated frequencies with averaged amino acid frequencies for mesophiles, $R_{\mathbf{M}}$, and thermophiles, $R_{\mathbf{T}}$. Multiple temperature profiles for the mesophilic correlation $R_{\mathbf{M}}$ (a) and thermophilic correlation $R_{\mathbf{T}}$ (c), are plotted for a set of trade--off paramteres $w$ using proteome--wide archaea data. Correlation profiles yielding overall highest $R_{\mathbf{M}}$ ($w=0.05$) and $R_{\mathbf{T}}$ ($w=0.06$) are depicted with blue and red correspondingly, along with the dashed guiding--lines for the optimal temperatures correspondingly $T^*_{\mathbf{M}}=1.0$ p.u.  and $T^*_{\mathbf{T}}=1.4$ p.u. Simulations are also compared with the average amino acid composition (b), $R_{\mathbf{A}}$, and with the slopes of amino acid temperature trends (d), $R_{\mathbf{D}}$. Correlation profiles corresponding to the overall optimal trade--off parameter $w^*=0.05$ are plotted in blue, and the unified optimal temperature range $1.0=T^**_{\mathbf{M}}<T<T^**_{\mathbf{T}}=1.5$ is highlighted in yellow. Optimal temperature range cover highest $R_{\mathbf{A}}$ and $R_{\mathbf{D}}$ values.


\begin{figure}[h!]
	\centering
	\includegraphics{../Publication/exp_MTAD_cai_trop_arch_summary_Figure_4.pdf}
\end{figure}



\end{document}