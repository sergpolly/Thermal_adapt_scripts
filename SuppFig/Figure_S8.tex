\documentclass{report}

\usepackage{graphicx}
\usepackage[lofdepth,lotdepth]{subfig}
\usepackage[top=0.7in, bottom=0.2in, left=0.3in, right=0.3in]{geometry}
\renewcommand{\thesubfigure}{\Alph{subfigure}}
\usepackage[percent]{overpic}


\begin{document}



{\bf Figure S8. Average cost of amino acid biosynthesis is negatively correlated with CAI.} Proteins in each organism are grouped into 5 quantiles according to their CAI rank and amino acid composition is averaged within each group. Trends of average biosynthesis costs  with expression level (CAI) are analyzed. Both archaeal ({\bf A}) and bacterial ({\bf B}) trends (red) demonstrate significant negative correlation, in accordance with earlier research by Akashi and Gojobori (see main text for details). Codon reshuffling distrupts protein--level CAI, while keepign genome-wide codon usage and GC intact. Using codon reshuffling we generated a control trend of biosynthesis cost with CAI that demonstrates no significant correlation and thus indicated that the trend observed for original data is not accidental. Error bars represent the 30\% and 70\% percentiles of the underlying distrbutions (data not shown).


% {\bf $R_T$ subtle correlation with CAI} Proteins of each organism are divided into 5 quantiles according to their CAI rank and amino acid composition is averaged within each group, so that statistically averaged trend with expression level (CAI) can be conveniently analyzed. $R_T$ is a Pearson correlation coefficient between the average amino acid composition in each quantile and the thermophilic composition of amino acids, calculated used OGT>50C criteria. (A) archaeal and (B) bacterial trends are presented along with a random realization of the codon reshuffling procedure to emphasize statistical significance of the original trend (or lack of thereof). P-value is calculated using XXX number of random reshufflings. [CHECK ...]

% {\bf IVYWREL trends with CAI} Proteins of each organism are divided into 5 quantiles according to their CAI rank and amino acid composition is averaged within each group, so that statistically averaged trend with expression level (CAI) can be conveniently analyzed.
% IVYWREL metric is averaged for each quantile and (A) archaeal and (B) bacterial trends are presented along with a random realization of the codon reshuffling procedure. Original result can be expected by the random and can be explained using specificity of IVYWREL and the genetic code link. [CHECK ...]


% \begin{figure}[h!]
% 	\centering
% 	\subfloat[Subfigure 1 list of figures text][Archaeal data]{
% 		\includegraphics[width=0.5\textwidth]{../ArchNew/BOOTSTRAP/Akashi_arch_qunatile_trend_original_trop.png}
% 		\label{fig:subfig1}
% 	}
% 	\subfloat[Subfigure 2 list of figures text][Bacterial data]{
% 		\includegraphics[width=0.5\textwidth]{../BOOTSTRAPS/Akashi_bact_qunatile_trend_original_trop.png}
% 		\label{fig:subfig2}
% 	}
% \end{figure}





\begin{center}
% \begin{overpic}[grid,tics=20,width=\textwidth]{../Publication/Supp9_quintiles.png}
\begin{overpic}[width=\textwidth]{../Publication/Supp9_quintiles.png}
% \put(32,74){\includegraphics[scale=.3] {busstop.mps}}
\put(9,38){\LARGE{\bf A}}
\put(59,38){\LARGE{\bf B}}
	% \put(32,77){\llap{\scriptsize \colorbox{back}{Windm\"uhle}}}
% \put(28,63){\small\textcolor{red}{\ding{55}}}
% \put(17.5,11){\scriptsize\colorbox{back}{{\Pisymbol{ftsy}{65} Fr}}}
% \put(6.3,13){\colorbox{back}{{\Pisymbol{ftsy}{68}}}}
% \put(29.8,61.4){\color{blue}\vector(-1,-3){2}}
% \put(38.6,63){\color{blue}\vector(1,3){2}}
\end{overpic}
\end{center}


% \begin{figure}[h!]
% 	\centering
% 	\includegraphics[width=\textwidth]{../Publication/Supp8_quintiles.png}
% 	\label{fig:subfig1}
% \end{figure}




\end{document}


% ../BOOTSTRAPS/bact_Akashi.png
% ../ArchNew/BOOTSTRAP/arch_Akashi.png








