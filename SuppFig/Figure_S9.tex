\documentclass{report}

\usepackage{graphicx}
\usepackage[lofdepth,lotdepth]{subfig}
\usepackage[top=0.7in, bottom=0.2in, left=0.3in, right=0.3in]{geometry}
\usepackage[percent]{overpic}

\begin{document}



% {\bf Figure S8. Synthesis costs negatively correlated with CAI} Proteins of each organism are divided into 5 quantiles according to their CAI rank and amino acid composition is averaged within each group, so that statistically averaged trend with expression level (CAI) can be conveniently analyzed. ({bf _1}) archaeal and ({bf }) bacterial trends are presented along with a random realization of the codon reshuffling procedure to emphasize statistical significance of the original trend. P-value is calculated using XXX number of random reshufflings. 

% % {\bf $R_T$ subtle correlation with CAI} Proteins of each organism are divided into 5 quantiles according to their CAI rank and amino acid composition is averaged within each group, so that statistically averaged trend with expression level (CAI) can be conveniently analyzed. $R_T$ is a Pearson correlation coefficient between the average amino acid composition in each quantile and the thermophilic composition of amino acids, calculated used OGT>50C criteria. ({bf $}) archaeal and ({bf }) bacterial trends are presented along with a random realization of the codon reshuffling procedure to emphasize statistical significance of the original trend (or lack of thereof). P-value is calculated using XXX number of random reshufflings. [CHECK ...]


{\bf Figure S9. Similarity between proteins with high CAI and thermophilic proteins: distribution details.}
Proteins in each organism are grouped into 5 quintiles according to their CAI value, and amino acid composition within these groups is compared with the average thermophilic composition using Pearson correlation coefficient $R_T$, separately for archaea ({\bf A}), and bacteria ({\bf B}). Details of $R_T$ distribution within each quintile are demonstrated for archaea ({\bf A}), and bacteria ({\bf B}), right columns (red), and distributions' mean values are highlighted as well. Synonymous codon reshuffling, disrupting CAI ranking, is used as a control. Corresponding distributions are demonstrated for archaea ({\bf A}), and bacteria ({\bf B}), left columns (blue), along with the mean values.
Mean values of $R_T$ are increasing in the case of bacteria, in contrast with the archaeal data that demonstrate no significant trend. Mean values of corresponding controls are either demonstrate no sigificant trends, or demonatrate negative trend, and thus strongly indicate the importance of original codon ranking and codon usage bias.




\begin{center}
\begin{overpic}[width=0.9\textwidth]{../ArchNew/BOOTSTRAP/RT_underhood_distro.png}
\put(-1,99){\LARGE{\bf A} (CUS Archaea)}
\end{overpic}
\end{center}


\begin{center}
\begin{overpic}[width=0.9\textwidth]{../BOOTSTRAPS/RT_underhood_distro.png}
\put(-1,99){\LARGE{\bf B} (CUS Bacteria)}
\end{overpic}
\end{center}





\end{document}


% ../BOOTSTRAPS/bact_Akashi.png
% ../ArchNew/BOOTSTRAP/arch_Akashi.png








