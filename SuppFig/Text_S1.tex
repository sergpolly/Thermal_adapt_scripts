\documentclass{report}

\usepackage{graphicx}
% \usepackage[lofdepth,lotdepth]{subfig}
% \usepackage[top=0.5in, bottom=0.5in, left=0.5in, right=0.8in]{geometry}
\usepackage[top=0.5in, bottom=0.5in, left=0.5in, right=0.5in]{geometry}

% textcomp package and marvosym package for additional characters
\usepackage{textcomp,marvosym}
\usepackage[percent]{overpic}


\usepackage{color,soul}
\sethlcolor{yellow}

\begin{document}


{\bf Text S1. Organismal and protein--level bootstraps to test the convergence of archaeal and bacterial thermal adaptation }

\begin{figure}[h!]
	\centering
	\includegraphics[width=0.5\textwidth]{../BOOTSTRAPS/SuppS1_ribo_test.png}
	\caption{
	{\bf Bootstrapping proteins for all analyzed organisms.} Distribution of correlation coefficient that reflects similarity between archaeal and bacterial temperature trends of amino acid compositions in the random protein sampling bootstrap. Original correlation coefficient R$\approx$0.73 for ribosomal proteins (red) is significantly larger than the bootstrapped correlations, p\textless0.001. 
	}
	\label{fig1}
\end{figure}

({\bf I})  Consideration of thermal adaptation only in ribosomal proteins yields similar results for archaea and bacteria ($R\approx0.7$), see main text Fig.2B. However, we must test the null hypothesis that the particular type of proteins does not matter, i.e., using any random sample of $\approx$50 proteins per genome, instead of ribosomal ones, would yield comparable correlation. We performed the bootstrap test using 100 random samples of 50 proteins per every archaeal and bacterial genome. Organism--average amino acid compositions were calculated for every sample and temperature trends were derived for archaea and bacteria. Pearson correlation coefficient $R$ was used as a measure of similarity between archaeal and bacterial trends. Histogramm of $R$ for all possible 100$\times$100 combinations between bootstrapped archaeal and bacteria trends is demonstrated, Fig.~\ref{fig1} (blue). The original correlation coefficient $R\approx0.7$ (red) is significantly higher than the bootstrapped values, $p<0.001$, thus we can safely reject the null hypothesis that the particular type of proteins does not matter.

\begin{figure}[h!]
	\centering
	\includegraphics[width=0.5\textwidth]{../BOOTSTRAPS/SuppS1_CUS_proteome_test.png}
	\caption{
	{\bf Bootstrapping organisms to test the proteome--wide trends of thermal adaptation in organisms with CUS.}
	Distribution of correlation coefficient that reflects similarity between archaeal and bacterial temperature trends of amino acid compositions in the random organism sampling bootstrap. Original correlation coefficient R$\approx$0.53 for organisms with CUS (red) is well within the distribution, p$\approx$0.26.
	}
	\label{fig2}
\end{figure}


({\bf II})  Proteome--wide thermal adaptation in the organisms with codon usage selection (CUS) yields significant correlation between archaeal and bacterial counterparts, R$\approx$0.55, see Fig.2C in the main text. However, we must test the null hypothesis that the particular type of organisms does not matter. There are $\sim$40\% of organisms with CUS both in archaeal and bacterial datasets, thus we performed a  bootstrap test by randomly selecting 40\% of organisms from each dataset for 100 samples. Proteome--wide compositions of amino acids were calculated for every organism in each sample, and then used to derive archaeal and bacterial temperature--trends of amino acid compositions per sample. Pearson correlation coefficient $R$ was used as a measure of similarity between archaeal and bacterial trends. Histogramm of $R$ for all possible 100$\times$100 combinations between bootstrapped archaeal and bacteria trends is demonstrated, Fig.~\ref{fig2} (blue). The original correlation coefficient $R\approx0.55$ (red) lies well within the distribution, p$\approx$0.26, and thus the null hypothesis cannot be rejected.



\begin{figure}[h!]
	\begin{center}
		\begin{overpic}[width=\textwidth]{../BOOTSTRAPS/SuppS1_PHX_test.png}
		\put(6,32){\LARGE{\bf A}}
		\put(56,32){\LARGE{\bf B}}
		\end{overpic}
	\caption{
		{\bf Bootstrapping organisms and proteins to test the trends in PHX proteins from organisms with CUS.}
		Distribution of correlation coefficient that reflects similarity between archaeal and bacterial temperature trends of amino acid compositions ({\bf A}) in organism sampling bootstrap using PHX proteins and ({\bf B}) in protein sampling bootstrap using organisms with CUS.	Original correlation coefficient R$\approx$0.87 (red) is higher than bootstrapped values in both cases, $p=0.026$ and $p<0.001$, correspondingly.
	}
	\label{fig3}
	\end{center}
\end{figure}


({\bf III}) We also tested observations for the predicted highly expressed proteins (PHX) in organisms with CUS, where PHX are defined as proteins with the CAI above 90{\it th} percentile of genome--wide CAI distribution. First, we tested the importance of organisms CUS for the similarity between archaeal and bacterial temperature trends of PHX proteins. Bootstrap (II) was repeated with the only difference that PHX proteins were used instead of full proteomes. Resulted histogram of $10^4$ values of correlation coefficients demonstrated, Fig.~\ref{fig3}(A) (blue), where the original observed $R\approx0.87$ (red) is significantly higher than the bootstrapped values, $p=0.026$, thus we can safely reject the null hypothesis that the particular choice of organisms does not matter for thermal trends in PHX proteins. 

Another null hypothesis for PHX proteins is that, given the organisms with CUS, the particular choice of proteins does not matter. To test it, we performed protein--bootstrap similar to (I) only using organisms with CUS and  randomly selecting 10\% of all proteins per genome. Resulted histogram of $10^4$ values (100 samples both for archaea and bacteria) of correlation coefficients demonstrated, Fig.~\ref{fig3}(B) (blue), where the original observed $R\approx0.87$ (red) is significantly higher than the bootstrapped values, $p<0.001$, thus we can safely reject the null hypothesis.

Therefore, similarity in thermal adaptation of PHX proteins from archaea and bacteria with CUS is a statistically significant results, unlike the proteome--wide adaptation in organisms with CUS. Similarity in thermal adaptation of ribosomal proteins is a statistically significant result as well.



\end{document}





