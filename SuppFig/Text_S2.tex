\documentclass{report}

\usepackage{graphicx}
\usepackage[lofdepth,lotdepth]{subfig}
\usepackage[top=0.5in, bottom=0.5in, left=0.5in, right=0.8in]{geometry}

\usepackage{color,soul}
\sethlcolor{yellow}


\begin{document}

% {\bf Validation of the simulated data using amino acid maintenance rate vector reshuffling.} Amino acid maintenance rate vector $C_{a}$ was reshuffled 100 times and used in the protein design procedure at $\mathit{w}_{op}$ and $T=0.9$. Resulted amino acid frequencies were compared with the evolved average amino acid frequencies, vector $\mathbf{A}$, using Pearson correlation coefficient $R_{\mathbf{A}}$. Histogram of thus simulated $R_{\mathbf{A}}$ values displayed on the figure, $R_{\mathbf{A}}$ value obtained original vector $C_{a}$ depicted as red dot.
% }
% %%%%%%%%%%%%%%%%%%%%%%%%%%%%%%%%%%%%%%%%%%%%%%%%%%
%  MOVE TO SUPPLEMENTARY TEXT ...
% %%%%%%%%%%%%%%%%%%%%%%%%%%%%%%%%%%%%%%%%%%%%%%%%%%
{\bf Text S2. Statistical validation of the simulations}
% %%%%%%%%%%%%%%%%%%%%%%%%%%%%%%%%%%%%%%%%%%%%%%%%%%
Values of the amino acid maintenance rate $C_{a}$ are crucial for explanation of amino acid frequencies and their temperature-trends in the naturally evolved species. We performed $100$ reshufflings of the maintenance rate vector $C_{a}$, and used these modified vectors to design proteomes according to the procedure used in the above analysis and described in {\bf Model} section. Design procedure was performed for a range of balance parameters $0.0\leq\mathit{w}\leq0.12$ and an artificial temperature $T=0.9$ corresponding to the optimal temperature, $T_{\mathbf{A}}$, explaining average amino acid frequencies over the entire dataset of evolved species. Simulated frequencies of amino acids are compared with the average amino acid compositions both in archaeal and bacterial species, using Pearson correlation coefficient $R_{\mathbf{A}}$. 

\begin{figure}[h!]
	\centering
	\subfloat[Subfigure 1 list of figures text][Archaeal data]{
		\includegraphics[width=0.9\textwidth]{../Publication/exp_MTAD_all_all_arch_summary_SuppFig3.pdf}
		\label{fig2s1}
	}
	\qquad
	\subfloat[Subfigure 2 list of figures text][Bacterial data]{
		\includegraphics[width=0.9\textwidth]{../Publication/exp_MTAD_all_all_bact_summary_SuppFig3.pdf}
		\label{fig2s2}
	}
	\caption{
	{\bf Validation of the simulated data using amino acid maintenance rate vector reshuffling.}(A) Validation using simulations at $T=0.9$ and the balance parameter $w$ ranging from 0.00 to 0.12. Reshuffling at $w=0.0$ does not alter the correlation (as expected), while increasing $w$ makes orginial results stand out significantly from the shuffled ones. (B) Shuffling the original cost vector disrupts the whole concept of the optimal balance parameter, as each of the 100 “trajectories” has its own optimal $w$, and these optimal $w$ distributed almost homogeneously over the entire $w$ range.
}
\end{figure}

As expected, metabolic rate $C_{a}$ vector reshuffling does not alter the outcome of the simulations solely based on the foldability requirement at $\mathit{w}=0.0$, where the maintenance rates do not impose any restrictions on the usage of amino acids, see Fig.~\ref{fig2s1}(archaea) and~\ref{fig2s2}(bacteria). At the same time, $R_{\mathbf{A}}$ designed with the original $C_{a}$ stands out significantly as $\mathit{w}$ growths, see Fig.~\ref{fig2s1} and~\ref{fig2s2}. We also extracted the balance parameter $\mathit{w}_{op}$ corresponding to the highest $R_{\mathbf{A}}$ for each one of the hundred reshuffles, and it is noteworthy, that the distribution of these $\mathit{w}_{op}$ is relatively uniform, see Fig.~\ref{fig2s1} and~\ref{fig2s2}, so that reshuffling obscures both the unified balance parameter $\mathit{w}_{op}\sim0.06$ specific for the original $C_{a}$ and the correlation with the naturally observed proteome level data.

Distribution of the resulted $R_{\mathbf{A}}$ values at the balance parameter value $\mathit{w}=0.06$ is presented on Fig.~\ref{fig1}(A), original (wild type) $R^{*}_{\mathbf{A}}$ is depicted in red and is significantly higher that the reshuffled values, \hl{pvalue}. We also used the reshuffling of the maintenance rate vector $C_{a}$ to confirm that temperature trends of amino acid frequencies rely on the original (wild type) vector $C_{a}$ for meaningfull outcomes of the protein design simulation. One hundred reshuffles of the $C_{a}$ were used to design model proteins within the range of temperatures close to the most plausible one $[T_\mathbf{M},T_\mathbf{T}]$, $1.0\leq T\leq1.4$ with the step $\Delta T=0.2$ and the balance parameter assuming two nearly optimal values $\mathit{w}_{T}=0.06$ and $\mathit{w}_{M}=0.06$. We used Pearson correlarion coefficient, $R_{\mathbf{D}}$, to relate the designed trends with the evolved one, vector $\mathbf{D}$. Distribution of the resulted $R_{\mathbf{D}}$ values at the balance parameter value $\mathit{w}_{M}=0.05$ is presented on Fig.~\ref{fig1}(B), original (wild type) $R^{*}_{\mathbf{D}}$ is depicted in red and is significantly higher that the reshuffled values, \hl{pvalue}. Performed validation implies that the agreement between our model and the observed compositional trends in amino acids is not accidental and the original values of the maintenance rate vector $C_{a}$ are highly non-random.



\begin{figure}[h!]
	\centering
	\subfloat[Subfigure 1 list of figures text][Archaeal data]{
		\includegraphics[width=0.9\textwidth]{../Publication/exp_MTAD_all_all_arch_summary_MainFig6.pdf}
		\label{fig2s1}
	}
	\qquad
	\subfloat[Subfigure 2 list of figures text][Bacterial data]{
		\includegraphics[width=0.9\textwidth]{../Publication/exp_MTAD_all_all_bact_summary_MainFig6.pdf}
		\label{fig2s2}
	}
	\caption{
	{\bf Validation of the simulated data using amino acid maintenance rate vector reshuffling.} Amino acid maintenance rate vector $C_{a}$ was reshuffled 100 times and used in the protein design procedure at $\mathit{w}_{op}$ and $T=0.9$. Resulted amino acid frequencies were compared with the evolved average amino acid frequencies, vector $\mathbf{A}$, using Pearson correlation coefficient $R_{\mathbf{A}}$. Histogram of thus simulated $R_{\mathbf{A}}$ values displayed on the figure, $R_{\mathbf{A}}$ value obtained original vector $C_{a}$ depicted as red dot.
}
\end{figure}





% \begin{figure}[h!]
% \includegraphics{../Publication/exp_MTAD_all_all_bact_summary_MainFig6.pdf}
% \caption{
% {\bf Validation of the simulated data using amino acid maintenance rate vector reshuffling.} Amino acid maintenance rate vector $C_{a}$ was reshuffled 100 times and used in the protein design procedure at $\mathit{w}_{op}$ and $T=0.9$. Resulted amino acid frequencies were compared with the evolved average amino acid frequencies, vector $\mathbf{A}$, using Pearson correlation coefficient $R_{\mathbf{A}}$. Histogram of thus simulated $R_{\mathbf{A}}$ values displayed on the figure, $R_{\mathbf{A}}$ value obtained original vector $C_{a}$ depicted as red dot.
% }
% % \label{fig:validation}
% \label{fig1}
% \end{figure}
%
%
% \begin{figure}[h!]
% \includegraphics{../Publication/exp_MTAD_all_all_arch_summary_MainFig6.pdf}
% \caption{
% {\bf Validation of the simulated data using amino acid maintenance rate vector reshuffling.} Amino acid maintenance rate vector $C_{a}$ was reshuffled 100 times and used in the protein design procedure at $\mathit{w}_{op}$ and $T=0.9$. Resulted amino acid frequencies were compared with the evolved average amino acid frequencies, vector $\mathbf{A}$, using Pearson correlation coefficient $R_{\mathbf{A}}$. Histogram of thus simulated $R_{\mathbf{A}}$ values displayed on the figure, $R_{\mathbf{A}}$ value obtained original vector $C_{a}$ depicted as red dot.
% }
% % \label{fig:validation}
% \label{fig3}
% \end{figure}


\end{document}



% ../Publication/exp_MTAD_all_all_arch_summary_SuppFig3.pdf
% ../Publication/exp_MTAD_all_all_bact_summary_SuppFig3.pdf














