\documentclass{report}

\usepackage{graphicx}
\usepackage[lofdepth,lotdepth]{subfig}
\usepackage[top=0.3in, bottom=0.5in, left=0.5in, right=0.5in]{geometry}

\usepackage{amsopn}

% textcomp package and marvosym package for additional characters
\usepackage{textcomp,marvosym}
\usepackage[percent]{overpic}

\usepackage{color,soul}
\sethlcolor{yellow}

\DeclareMathOperator*{\argmax}{arg\,max}

\begin{document}

% %%%%%%%%%%%%%%%%%%%%%%%%%%%%%%%%%%%%%%%%%%%%%%%%%%
{\bf Text S2. Statistical validation of the simulations.}
% %%%%%%%%%%%%%%%%%%%%%%%%%%%%%%%%%%%%%%%%%%%%%%%%%%
Values of the amino acid maintenance cost $C_{a}$ are crucial for explanation of amino acid frequencies and their temperature--trends. We performed $100$ reshufflings of the cost vector $C_{a}$, and used these modified vectors to design proteomes according to the procedure described in {\bf Model} section, main text. The design procedure was performed for a range of cost adjustment parameters $0.0\leq w\leq0.12$ and temperature $T=0.9$, which lies with optimal $T$-ranges predicted both for archaea and bacteria. Pearson correlation coefficient is used to compare amino acid frequencies from simulated proteomes with the average composition of amino acids $R_A=R_{M\cup T}$ both in archaeal and bacterial species.


\begin{figure}[h!]
	\begin{center}
		\begin{overpic}[width=0.9\textwidth]{../Publication/SuppText2_Fig1.png}
		\put(11,86){\LARGE{\bf A}}
		\put(60,86){\LARGE{\bf B} Archaea}
		\put(11,40.5){\LARGE{\bf C}}
		\put(60,40.5){\LARGE{\bf D} Bacteria}
		%%%%%%%%%%%%%%%%%%%%%%%%%%%%
		\put(33,86.5){\large original $R_A(w)$}
		\put(25,40){\large original $R_A(w)$}
		\end{overpic}
	\caption{
		{\bf Validation of the simulated data using amino acid maintenance rate vector reshuffling.} ({\bf A,C})
		Correlation coefficient, $R_A(w)$, between observed composition of amino acids and frequencies derived from simulated proteomes is plotted as a function of adjustment parameter $w$. Proteomes are simulated at $T=0.9$ using reshuffled costs $C_{a}$, and compared with ({\bf A}) archaeal data and ({\bf C}) bacterial data. Each $R_A(w)$ is colored--coded according to its optimal $\displaystyle w^*=\argmax_{w} R_A(w)$, see legend. $R_A(w)$ corresponding to the original $C_a$ is labeled as well.
		({\bf B,D}) $R_A(w)$ corresponding to different shuffle of $C_a$ yields differenct optimal values of adjustment parameter $w^*$, disrupting the ``concept'' of optimal $w^*$. Distribution of such $w^*$ for ({\bf B}) archaea and ({\bf D}) bacteria are plotted.
	}
	\label{fig1}
	\end{center}
\end{figure}

As expected, reshuffling of the maintenance cost vector $C_a$ does not alter the outcome of the simulations for $w=0$, where the maintenance cost does not impose any restrictions on the usage of amino acids, Fig~\ref{fig1}(A,C). Simulations with the original cost vector $C_a$ yields correlation $R_A(w)$ that stands out significantly as $w>0.02$ increases, Fig.~\ref{fig1}(A,C).
Cost adjustment parameter $w^*$ determined for 100 cost shuffles is concentrated near $w=0.0$ for archaea, Fig~\ref{fig1}(B), and is relatively homogeneous in the case of bacteria, Fig~\ref{fig1}(D). It implies either that lack of maintenance costs yields better predictions for proteome--average composition or that the concept of the unified balance parameter $w^*\sim0.06$ is disrupted. These observations emphasize specificity of the original maintenance cost vector $C_a$.


\begin{figure}[h!]
	\begin{center}
		\begin{overpic}[width=0.9\textwidth]{../Publication/SuppText2_Fig2.png}
		\put(8,85){\LARGE{\bf A} Archaea}
		\put(53,85){\LARGE{\bf B}}
		\put(8,40){\LARGE{\bf C} Bacteria}
		\put(53,40){\LARGE{\bf D}}
		\end{overpic}
	\caption{
		{\bf Validation of the simulated data using amino acid maintenance cost $C_a$ reshuffling.} Amino acid maintenance rate vector $C_{a}$ was reshuffled 100 times and used to simulate proteomes at $w^*=0.06$ and $T=0.9$. Compositions of the simulated proteomes were compared with the observed composition of amino acids using correlation coefficient $R_A$ for ({\bf A}) archaea (blue), and ({\bf C}) bacteria (blue). Values of $R_A$ corresponding to the original vector $C_a$ (red) are significantly higher than the values corresponding to shuffled $C_a$, p\textless0.001 in both cases.
		Trends of thermal adaptation were simulated using reshuffled cost vector $C_a$ in $T$-range, $1.0\leq T\leq1.4$, and $w^*=0.06$. Simulated trends were compared with the observed trends for ({\bf B}) archaea (blue), and ({\bf D}) bacteria (blue). Values of $R_D$ corresponding to the original vector $C_a$ (red) are significantly higher than the values corresponding to shuffled $C_a$, p\textless0.001 and p=0.06 for archaea and bacteria, correspondingly.
	}
	\label{fig2}
	\end{center}
\end{figure}


Distribution of the resulted $R_A$ values at a balance parameter $w=0.06$ is presented, Fig.~\ref{fig2}(A) for archaea (blue), and Fig.~\ref{fig2}(C) for bacteria (blue). Values of $R_A(w=0.06)$ predicted for original maintenance cost vector $C_a$, are depicted as red bars on both plots, Fig.~\ref{fig2}(A,C), and both are significantly higher than the $R_A$ corresponding to shuffled vector $C_a$, p\textless0.001 in both cases. We also tested how reshuffling of the maintenance cost $C_a$ affects correlation of slopes of thermal adaptation $df/dT$ between observed and simulated data, $R_D$. One hundred reshuffles of the cost vector $C_{a}$ were used for design of model proteomes within the range of temperatures $1.0\leq T\leq1.4$, with the step of $\Delta T=0.2$, and the balance parameter $w^*=0.06$. Assumed parameters are close to their optimal values, especially for archaeal results, see Fig 4, main text and Fig S5. Distribution of the resulted values of Pearson correlarion coefficient $R_D$ is demonstrated, Fig.~\ref{fig2}(B) for archaea, and Fig.~\ref{fig2}(D) for bacteria. Values of $R_D$ corresponding to the original vector $C_a$ are depicted in red, Fig.~\ref{fgi2}(B,D), and both of them are significantly higher than the $R_D$ corresponding to shuffled $C_a$, p\textless0.001 for archaea and p=0.06 for bacteria.
These validation simualtions strongly indicate that the agreement between our model and the observed compositional trends in amino acids is not accidental and the original values of the maintenance cost vector $C_{a}$ are highly non-random, and must be used in conjunction with the physical interaction energy $\epsilon_{i,j}$ for a given type of amino acid.



\end{document}



% ../Publication/exp_MTAD_all_all_arch_summary_SuppFig3.pdf
% ../Publication/exp_MTAD_all_all_bact_summary_SuppFig3.pdf














